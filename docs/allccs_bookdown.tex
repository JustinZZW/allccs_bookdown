\documentclass[12pt,]{book}
\usepackage{lmodern}
\usepackage{amssymb,amsmath}
\usepackage{ifxetex,ifluatex}
\usepackage{fixltx2e} % provides \textsubscript
\ifnum 0\ifxetex 1\fi\ifluatex 1\fi=0 % if pdftex
  \usepackage[T1]{fontenc}
  \usepackage[utf8]{inputenc}
\else % if luatex or xelatex
  \ifxetex
    \usepackage{mathspec}
  \else
    \usepackage{fontspec}
  \fi
  \defaultfontfeatures{Ligatures=TeX,Scale=MatchLowercase}
\fi
% use upquote if available, for straight quotes in verbatim environments
\IfFileExists{upquote.sty}{\usepackage{upquote}}{}
% use microtype if available
\IfFileExists{microtype.sty}{%
\usepackage{microtype}
\UseMicrotypeSet[protrusion]{basicmath} % disable protrusion for tt fonts
}{}
\usepackage[margin=1in]{geometry}
\usepackage{hyperref}
\PassOptionsToPackage{usenames,dvipsnames}{color} % color is loaded by hyperref
\hypersetup{unicode=true,
            pdftitle={AllCCS Tutorial V1.00},
            pdfauthor={Zhiwei Zhou},
            colorlinks=true,
            linkcolor=Maroon,
            citecolor=Blue,
            urlcolor=Blue,
            breaklinks=true}
\urlstyle{same}  % don't use monospace font for urls
\usepackage{natbib}
\bibliographystyle{apalike}
\usepackage{longtable,booktabs}
\usepackage{graphicx,grffile}
\makeatletter
\def\maxwidth{\ifdim\Gin@nat@width>\linewidth\linewidth\else\Gin@nat@width\fi}
\def\maxheight{\ifdim\Gin@nat@height>\textheight\textheight\else\Gin@nat@height\fi}
\makeatother
% Scale images if necessary, so that they will not overflow the page
% margins by default, and it is still possible to overwrite the defaults
% using explicit options in \includegraphics[width, height, ...]{}
\setkeys{Gin}{width=\maxwidth,height=\maxheight,keepaspectratio}
\IfFileExists{parskip.sty}{%
\usepackage{parskip}
}{% else
\setlength{\parindent}{0pt}
\setlength{\parskip}{6pt plus 2pt minus 1pt}
}
\setlength{\emergencystretch}{3em}  % prevent overfull lines
\providecommand{\tightlist}{%
  \setlength{\itemsep}{0pt}\setlength{\parskip}{0pt}}
\setcounter{secnumdepth}{5}
% Redefines (sub)paragraphs to behave more like sections
\ifx\paragraph\undefined\else
\let\oldparagraph\paragraph
\renewcommand{\paragraph}[1]{\oldparagraph{#1}\mbox{}}
\fi
\ifx\subparagraph\undefined\else
\let\oldsubparagraph\subparagraph
\renewcommand{\subparagraph}[1]{\oldsubparagraph{#1}\mbox{}}
\fi

%%% Use protect on footnotes to avoid problems with footnotes in titles
\let\rmarkdownfootnote\footnote%
\def\footnote{\protect\rmarkdownfootnote}

%%% Change title format to be more compact
\usepackage{titling}

% Create subtitle command for use in maketitle
\newcommand{\subtitle}[1]{
  \posttitle{
    \begin{center}\large#1\end{center}
    }
}

\setlength{\droptitle}{-2em}

  \title{AllCCS Tutorial V1.00}
    \pretitle{\vspace{\droptitle}\centering\huge}
  \posttitle{\par}
    \author{Zhiwei Zhou}
    \preauthor{\centering\large\emph}
  \postauthor{\par}
      \predate{\centering\large\emph}
  \postdate{\par}
    \date{2019-11-01}

\usepackage{booktabs}

\begin{document}
\maketitle

{
\hypersetup{linkcolor=black}
\setcounter{tocdepth}{1}
\tableofcontents
}
\listoftables
\listoffigures
\chapter*{About AllCCS}\label{about-allccs}
\addcontentsline{toc}{chapter}{About AllCCS}

Copyright (c) 2019 AllCCS Development Team

\href{http://allccs.zhulab.cn/}{\textbf{AllCCS}} is a powerful platform
to support various applications in Ion Mobility -- Mass Spectrometry
(IM-MS). It is designed to contain three major parts: \textbf{1) Unified
CCS database, 2) Machine learning based CCS prediction, and 3) small
molecule annotation}. The unified CCS database is one of the most
comprehensive CCS databases, covering \textasciitilde{}1,700,000 small
molecules. It provides a universal platform to contain both experimental
CCS values (3,539) and predicted CCS values (over 10,000,000). Machine
learning based CCS prediction function supports convenient prediction
from SMILES structure to CCS values. This function utilizes the second
generation CCS prediction algorithm to generate CCS values and RSS score
for novel structures. Small molecule annotation provides an easy-to-use
annotation function for various features or compounds. It is supported
to search database with measured m/z and CCS for annotation, or in
conjunct with any other annotation tools, such as MetFrag, CFM-ID,
MS-Finder, and SIRUS etc.

Zhiwei Zhou (\url{zhouzw@sioc.ac.cn}) Zheng-Jiang Zhu
(\url{jiangzhu@sioc.ac.cn}) \href{http://www.zhulab.cn/}{Laboratory for
Mass Spectrometry and Metabolomics}
\href{http://www.ircbc.ac.cn/}{IRCBC}, Shanghai Institute of Organic
Chemistry Chinese Academy of Sciences, Shanghai, China

\chapter*{Citation}\label{citation}
\addcontentsline{toc}{chapter}{Citation}

If AllCCS is useful in your project, please cite our articles.

\begin{itemize}
\tightlist
\item
  Z. Zhou, Z.-J. Zhu* etc. Advancing CCS database towards metabolite
  annotation, In preparing
\end{itemize}

\chapter{Introduction of AllCCS}\label{intro}

AllCCS is a powerful platform to support various applications in Ion
Mobility -- Mass Spectrometry (IM-MS). It is designed to contain three
major parts: 1) Unified CCS database, 2) Machine learning based CCS
prediction, and 3) small molecule annotation. The unified CCS database
is one of the most comprehensive CCS databases, covering
\textasciitilde{}1,700,000 small molecules. It provides a universal
platform to contain both experimental CCS values (3,539) and predicted
CCS values (over 10,000,000). Machine learning based CCS prediction
function supports convenient prediction from SMILES structure to CCS
values. This function utilizes the second generation CCS prediction
algorithm to generate CCS values and RSS score for novel structures.
Small molecule annotation provides an easy-to-use annotation function
for various features or compounds. It is supported to search database
with measured m/z and CCS for annotation, or in conjunct with any other
annotation tools, such as MetFrag, CFM-ID, MS-Finder, and SIRUS etc.

\chapter{Quick Start Guide}\label{quick-start-guide}

Here is Quick Start Guide

\chapter{CCS Database}\label{ccs-database}

\section{Compound Browser}\label{compound-browser}

\section{Compound Card}\label{compound-card}

\section{Advanced Search}\label{advanced-search}

\chapter{CCS Prediction}\label{ccs-prediction}

\chapter{Metabolite Annotation}\label{metabolite-annotation}

We have finished a nice book.

\bibliography{book.bib,packages.bib}


\end{document}
